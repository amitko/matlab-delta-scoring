\documentclass[12pt]{article}
\usepackage[utf8]{inputenc}
%\usepackage[bulgarian]{babel}
\usepackage[pdftex]{graphicx}
\usepackage{matlab-prettifier}

\title{DELTA SCORING FUNCTION REFERENCE}
%\subtitle{version 0.1}
\author{D. Atanasov}

\begin{document}

\maketitle

\section{Introduction}
This document describes the set of functions and the way of their ussage for obtaining the DELTA SCORING approach for test evaluations.

\section{Instalation}
To install the package download it from \\
{\tt https://github.com/amitko/matlab-delta-scoring.git}. 
\\
Place the folder in the MATLAB path and rename it to {\tt +deltaScoring}.

\section{Ussage}
Here is an example of a simple ussage of the package.

Suppose the raw dichotomous item response is placed in variable {\tt itemScore}.

To estimate the expected item difficulties ("deltas") the bootstrapping procedure

\begin{lstlisting}[style=Matlab-bw]
[ItemDelta, estimatedDeltaSE] = ...
	deltaScoring.estimate.itemDeltaBootstrap(itemScore);
\end{lstlisting}

is called. The resulted item deltas and the corresponding standard error of estimate are returned in variables {\tt ItemDelta} and  {\tt estimatedDeltaSE}.

The classical person D-scores are calculated using the response paterns in itemScore and already calculated item deltas.

\begin{lstlisting}[style=Matlab-bw]
personDscores = ...
	deltaScoring.scoring.dScore(ItemDelta,itemScore,opt.Dscore_method);
\end{lstlisting}

Here {\tt opt} is a structure containing the options for the considered delta scoring model. It can be generated by

\begin{lstlisting}[style=Matlab-bw]
opt = deltaScoring.scoring.Options;
\end{lstlisting}

Here and after the default will be the model RFM2. If a RFM3 model is required this can be stated in 

\begin{lstlisting}[style=Matlab-bw]
opt.model = 3;
\end{lstlisting}

and the corresonding options should be passed to the functions.

The item parameters location $b$ and shape $s$ can be obtained by

\begin{lstlisting}[style=Matlab-bw]
[params, CI, ~, Results] = ...
deltaScoring.estimate.logitDeltaFit(itemScore,personDscores,opt);
\end{lstlisting}

where {\tt params} contains the matrix with corresponding parameters for any item in the test $[b,s]$. The first column corresponds to the location paramater  $b$ while the second represents the shape $s$. If the model is RMF3, the guessing parameter is in the third column.

The matrix {\tt CI} contains the 95\% confidence interval of the estimated values. {\tt Results} contains additional fitting parameters (for example MAD is available in {\tt Results.MAD}).

The persons true scores can be calculated by
\begin{lstlisting}[style=Matlab-bw]
personTrueScores = ...
deltaScoring.scoring.trueScore(ItemDelta,ItemParameters,personDscores,opt);
\end{lstlisting}

and the SE 

\begin{lstlisting}[style=Matlab-bw]
personTrueScoresSE = ...
deltaScoring.scoring.trueScoreSE(ItemDelta,ItemParameters,personDscores,opt);
\end{lstlisting}


A latent verssion of the location and shape parameters (together with their SE) can be obtained by 
\begin{lstlisting}[style=Matlab-bw]
[LatentParams,LatentSE] =  deltaScoring.estimate.ML_RFM_params( itemScore, personDscores, opt);
\end{lstlisting}

The corresponding MAD is obtained by
\begin{lstlisting}[style=Matlab-bw]
LatentMAD = deltaScoring.item.MAD(LatentParams, Results.observedLogitDelta ,opt);
\end{lstlisting}

where {\tt Results.observedLogitDelta} is calculated with \\{\tt deltaScoring.estimate.logitDeltaFit} above and contains the proportion of observed correct answers for the values on the D-score scale.

The corresponding latent values of the person D-scores ( and SE) can be obtained by
\begin{lstlisting}[style=Matlab-bw]
[latentScore, ~, latentScoreSE]= ...
deltaScoring.estimate.ML_RFM_scores( itemScore,LatentParams, opt)
\end{lstlisting}

The estimations of latent item parameters and person D-scores can be iterated untill a convergence is reached

\begin{lstlisting}[style=Matlab-bw]
OldLatentParams = zeros(size(LatentParams));
while max(abs(OldLatentParams - LatentParams)) > eps
	OldLatentParams = LatentParams;
	[LatentParams,LatentSE] =  ...
		deltaScoring.estimate.ML_RFM_params( itemScore, personDscores, opt);
	[latentScore, ~, latentScoreSE]= ...
		deltaScoring.estimate.ML_RFM_scores( itemScore,LatentParams, opt)
end
\end{lstlisting}

Equating of different tests can be performed by functions located in {\tt deltaScoring.equating}. Here an example of equating of classical (nonlatent) parameters of the test will be presented. The equating is based of caclulation of two constants $A$ and $B$ which represent the change of scale for the test equating. These constants can be calculated on the base of  item deltas of the target test {\tt  targetDeltas}, item deltas (allready colculated above) and few common items between the two tests (stated in the variable {\tt CommonItems} )

\begin{lstlisting}[style=Matlab-bw]
   [A,B] = deltaScoring.equating.constants(targetDeltas, ItemDelta, CommonItems);
   equatedItemDeltas = deltaScoring.equating.rescale(ItemDelta,A,B);
   equatedDscores = deltaScoring.scoring.dScore(ItemDelta,itemScore,opt.Dscore_method);
\end{lstlisting}


\section{Function reference}

%%%%%%%%%%%%%
\par\noindent\rule{\textwidth}{0.4pt}
{\bf\tt deltaScoring.assembly.multipleTest}
\par\noindent\rule{\textwidth}{0.4pt}
\begin{lstlisting}[style=Matlab-bw]
 multipleTest(nOfTest, nOfItems, itemDeltas, varargin)

 Returnt the item idexes from the itemParams
 which compose a number of tests test.

 nOfTest
 nOfItems - number of items in teh test.
 itemDeltas - list (column) of estimated item deltas

 Optional parameters: ['Name',value] pairs
 The approach is based on Linear Optimal Test Design
 Uses singleTest.
\end{lstlisting}
%%%%%%%%%%%%%
\par\noindent\rule{\textwidth}{0.4pt}
{\bf\tt deltaScoring.assembly.singleTest}
\par\noindent\rule{\textwidth}{0.4pt}
\begin{lstlisting}[style=Matlab-bw]

 Returns the item idexes from the itemParams
 which compose a test.
 
 nOfItems - number of items in teh test.
 itemDeltas - list (column) of estimated item deltas

 Optional parameters: ['Name',value] pairs
 The approach is based on Linear Optimal Test Design
\end{lstlisting}
%%%%%%%%%%%%%
\par\noindent\rule{\textwidth}{0.4pt}
{\bf\tt deltaScoring.dif.conditionalDIF}
\par\noindent\rule{\textwidth}{0.4pt}
\begin{lstlisting}[style=Matlab-bw]
 conditionalDIF(focal_params,reference_params,o)

 probabilities for correct item performance between
 focal and reference group, based on parameters of 
 the items for the two groups.
 
 o - deltaScoring.scoring.Options

\end{lstlisting}
%%%%%%%%%%%%%
\par\noindent\rule{\textwidth}{0.4pt}
{\bf\tt deltaScoring.dif.ESonDIF}
\par\noindent\rule{\textwidth}{0.4pt}
\begin{lstlisting}[style=Matlab-bw]
 [onFocal, onReference] = ESonDIF(focal_params,reference_params,o)
 Calculates the effect size on focal and reference group for DIF
 
 INPUT:
		focal_params 	 - item parameters, estimated on focal group
		reference_params - item parameters, estimated on reverence group
       o                - options
\end{lstlisting}
%%%%%%%%%%%%%
\par\noindent\rule{\textwidth}{0.4pt}
{\bf\tt deltaScoring.dif.functioning}
\par\noindent\rule{\textwidth}{0.4pt}
\begin{lstlisting}[style=Matlab-bw]
 Calculates different characteristics, corresponding to the DIF
\end{lstlisting}
%%%%%%%%%%%%%
\par\noindent\rule{\textwidth}{0.4pt}
{\bf\tt deltaScoring.dif.Mantel\_Haenszel}
\par\noindent\rule{\textwidth}{0.4pt}
\begin{lstlisting}[style=Matlab-bw]
 [a_MH,log_a_MH_SE,da_MH,z,p,MH,type,against]=Mantel_Haenszel(response,score,groups,reference)
 Calculates Mantel-Haenszel statistics
\end{lstlisting}
%%%%%%%%%%%%%
\par\noindent\rule{\textwidth}{0.4pt}
{\bf\tt deltaScoring.dif.MPDonDIF}
\par\noindent\rule{\textwidth}{0.4pt}
\begin{lstlisting}[style=Matlab-bw]
 [onFocal, onReference] = MPDonDIF(dif)
 dif = pcr_f - pcr_r;
\end{lstlisting}
%%%%%%%%%%%%%
\par\noindent\rule{\textwidth}{0.4pt}
{\bf\tt deltaScoring.dif.paramsForGroups}
\par\noindent\rule{\textwidth}{0.4pt}
\begin{lstlisting}[style=Matlab-bw]
 [focal_params,reference_params,deltasF,deltasR] = paramsForGroups(itemResponse,group,latent,o)
 Calcultes the item parameters for the focal and reference group

 group 0 - reference, 1 - focal
 params by default are latent
\end{lstlisting}
%%%%%%%%%%%%%
\par\noindent\rule{\textwidth}{0.4pt}
{\bf\tt deltaScoring.dif.plotConditionalDIF}
\par\noindent\rule{\textwidth}{0.4pt}
\begin{lstlisting}[style=Matlab-bw]
 plotConditionalDIF(dif,opt,visible)
 Plots conditional DIF for the Focal and reference group
\end{lstlisting}
%%%%%%%%%%%%%
\par\noindent\rule{\textwidth}{0.4pt}
{\bf\tt deltaScoring.dif.plotICC}
\par\noindent\rule{\textwidth}{0.4pt}
\begin{lstlisting}[style=Matlab-bw]
 plotICC(focal_params,reference_params,o,visible)
 Plots ICC for the Focal and reference group
\end{lstlisting}
%%%%%%%%%%%%%
\par\noindent\rule{\textwidth}{0.4pt}
{\bf\tt deltaScoring.dif.plotTSC}
\par\noindent\rule{\textwidth}{0.4pt}
\begin{lstlisting}[style=Matlab-bw]
 plotTSC(focal_params,reference_params,o,visible)
 Plots test score for the Focal and reference group
\end{lstlisting}
%%%%%%%%%%%%%
\par\noindent\rule{\textwidth}{0.4pt}
{\bf\tt deltaScoring.dif.testing}
\par\noindent\rule{\textwidth}{0.4pt}
\begin{lstlisting}[style=Matlab-bw]
 [DIF, DIFT,  HA, HB, HAT, HBT, Results] = testing(focal_params, reference_params, o)
 Tests DIF having estimated item parameters for focal
 and reference group.

 DIF - indicator for a specific item
       0 - no DIF
       1 - against focal
       2 - against reference
 DIFT - DIF on test level
 HA   - Hypothesis nder approach A. Cell array of structures
 HB   - Hypothesis nder approach B. Cell array of structures
 HAT on test level
 HBT on test level
 Results - structure with detailed results
\end{lstlisting}
%%%%%%%%%%%%%
\par\noindent\rule{\textwidth}{0.4pt}
{\bf\tt deltaScoring.equating.constants}
\par\noindent\rule{\textwidth}{0.4pt}
\begin{lstlisting}[style=Matlab-bw]
 [A,B] = constants(Base_test_deltas,New_test_deltas,common_items)
              Y        X
 Calculates the rescaling constants, based on common items
 between two test.

 INPUT:
	Base_test_deltas - item deltas of the base test
	New_test_deltas  - item deltas of the new test
   common_items     - twoo columns
				[base_test_item_id  new_test_item_id]

 OUTPUT:
		A and B
\end{lstlisting}
%%%%%%%%%%%%%
\par\noindent\rule{\textwidth}{0.4pt}
{\bf\tt deltaScoring.equating.dscore\_common}
\par\noindent\rule{\textwidth}{0.4pt}
\begin{lstlisting}[style=Matlab-bw]
 NOT IN USE
\end{lstlisting}
%%%%%%%%%%%%%
\par\noindent\rule{\textwidth}{0.4pt}
{\bf\tt deltaScoring.equating.dscore\_rfm}
\par\noindent\rule{\textwidth}{0.4pt}
\begin{lstlisting}[style=Matlab-bw]
 [Dscore_equated] = dscore_rfm(X_params, X_rescaled, Dscores, type, o)
 Calculates equated latent D-Score based on the latent parameters

 INPUT:
		X_params   - latent parameters of the test
       X_rescaled - rescaled parameters of the test after equating
		Dscores    - persons D-score
		type       - default value is m1
		o	       -  options

 OUTPUT:
		Dscore_equated - equated D-score
\end{lstlisting}
%%%%%%%%%%%%%
\par\noindent\rule{\textwidth}{0.4pt}
{\bf\tt deltaScoring.equating.dscoreOnSubtest}
\par\noindent\rule{\textwidth}{0.4pt}
\begin{lstlisting}[style=Matlab-bw]
 NOT IN USE
\end{lstlisting}
%%%%%%%%%%%%%
\par\noindent\rule{\textwidth}{0.4pt}
{\bf\tt deltaScoring.equating.rescale}
\par\noindent\rule{\textwidth}{0.4pt}
\begin{lstlisting}[style=Matlab-bw]
 res = rescale(deltas,A,B)
 Rescales the item deltas of a test according
 to rescaling constants A and B.
\end{lstlisting}
%%%%%%%%%%%%%
\par\noindent\rule{\textwidth}{0.4pt}
{\bf\tt deltaScoring.equating.rescale\_rfm}
\par\noindent\rule{\textwidth}{0.4pt}
\begin{lstlisting}[style=Matlab-bw]
 [X_params_rescaled, opts] = rescale_rfm(X_params,Y_params,common_items,method_type,o)
 Calculates the rescaled latent item parameters of test X on the scale of the base test Y.

 INPUT: 
		X_params - parameters of test X
		Y_params - parametres of test Y
		common_items - common items;  twoo columns 
				[base_test_item_id  new_test_item_id]
		method_type - Rescaling of the shape parameter s has two options
						direct [default] | trough_a 

 OUTPUT: 
		X_params_rescaled
		opts - Structure eith
				bA,bB,sA,sB
\end{lstlisting}
%%%%%%%%%%%%%
\par\noindent\rule{\textwidth}{0.4pt}
{\bf\tt deltaScoring.estimate.EM\_RFM}
\par\noindent\rule{\textwidth}{0.4pt}
\begin{lstlisting}[style=Matlab-bw]
  Function [pars,ability] = irt.ItemParametersEstimate_EM_3PL( data,o)
      estimates the parameters of the item characreristic
      curves under the IRT model usen the EM algorith.

  Input:
      data - Dihotomous item response
      o    - scoring.Options (optional)
  Output:
      pars - Item parapeters
           [difficulty, discrimination, guessing]
\end{lstlisting}
%%%%%%%%%%%%%
\par\noindent\rule{\textwidth}{0.4pt}
{\bf\tt deltaScoring.estimate.itemDeltaBootstrap}
\par\noindent\rule{\textwidth}{0.4pt}
\begin{lstlisting}[style=Matlab-bw]
\end{lstlisting}
%%%%%%%%%%%%%
\par\noindent\rule{\textwidth}{0.4pt}
{\bf\tt deltaScoring.estimate.latentLklh}
\par\noindent\rule{\textwidth}{0.4pt}
\begin{lstlisting}[style=Matlab-bw]
 latentLklh(xi,itemResponse,deltaScores,o)
 Calculates person likelihood on a specific test with a specific item response
 For internal use in estimations
\end{lstlisting}
%%%%%%%%%%%%%
\par\noindent\rule{\textwidth}{0.4pt}
{\bf\tt deltaScoring.estimate.logitDeltaFit}
\par\noindent\rule{\textwidth}{0.4pt}
\begin{lstlisting}[style=Matlab-bw]
\end{lstlisting}
%%%%%%%%%%%%%
\par\noindent\rule{\textwidth}{0.4pt}
{\bf\tt deltaScoring.estimate.logitDeltaPlot}
\par\noindent\rule{\textwidth}{0.4pt}
\begin{lstlisting}[style=Matlab-bw]
 h = logitDeltaPlot(GF,observedLogitDelta,o)
 Plots the fit and the estimated logistics curve
 Returns the figure object

 INPUT:
    GF - output from logitDeltaFit
    observedLogitDelta - from Results of logitDeltaFit
    o - options
           dScale
\end{lstlisting}
%%%%%%%%%%%%%
\par\noindent\rule{\textwidth}{0.4pt}
{\bf\tt deltaScoring.estimate.ML\_RFM\_params}
\par\noindent\rule{\textwidth}{0.4pt}
\begin{lstlisting}[style=Matlab-bw]
 [pars,se] = ML_RFM_params( itemResponse, deltaScores, o)
 Estimates the latent parameters of the items base on 
 RFM model, using JML approach.

 INPUT: 
		itemResponse - dichotomous item response
		deltaScores  - person D-scores	
		o            - oprions 

 OUTPUT:
		pars         - estimated parameter values
		se           - standard errors of the estimates
\end{lstlisting}
%%%%%%%%%%%%%
\par\noindent\rule{\textwidth}{0.4pt}
{\bf\tt deltaScoring.estimate.ML\_RFM\_scores}
\par\noindent\rule{\textwidth}{0.4pt}
\begin{lstlisting}[style=Matlab-bw]
 [scores,se, see]=ML_RFM_scores( itemResponse, itemParams, o)
 Estimates the latent parameters for person abilities, based on
 RFM model, using JML approach.

 INPUT: 
		itemResponse - dichotomous item response
		itemParams  - person D-scores	
		o            - oprions 

 OUTPUT:
		scores       - person latent D-scores
		se           - standard errors of the estimates
       see          - analitical solution for se  
\end{lstlisting}
%%%%%%%%%%%%%
\par\noindent\rule{\textwidth}{0.4pt}
{\bf\tt deltaScoring.generate.guttman}
\par\noindent\rule{\textwidth}{0.4pt}
\begin{lstlisting}[style=Matlab-bw]
 guttman(NofPersons,itemParams,options,reverse)
 Generates a item response according Guttman concept
\end{lstlisting}
%%%%%%%%%%%%%
\par\noindent\rule{\textwidth}{0.4pt}
{\bf\tt deltaScoring.generate.itemResponse}
\par\noindent\rule{\textwidth}{0.4pt}
\begin{lstlisting}[style=Matlab-bw]
 [res, out] = itemResponse(Persons,itemParams,options,env)
 Generates an item response Persons over a set
 of items, defined by their item Parameters.

 env is a structure containing additional
 information about cheating and guessing.
\end{lstlisting}
%%%%%%%%%%%%%
\par\noindent\rule{\textwidth}{0.4pt}
{\bf\tt deltaScoring.item.characteristicsFromParameters}
\par\noindent\rule{\textwidth}{0.4pt}
\begin{lstlisting}[style=Matlab-bw]
 res = characteristicsFromParameters(item_params,o)
 Calculates the item characteristics from item parameters

 INPUT: 
	item_params - item parameters
	o           - options

 OUTPUT
	res - [location, discrimination]	
\end{lstlisting}
%%%%%%%%%%%%%
\par\noindent\rule{\textwidth}{0.4pt}
{\bf\tt deltaScoring.item.icc}
\par\noindent\rule{\textwidth}{0.4pt}
\begin{lstlisting}[style=Matlab-bw]
 res = icc(itemParameters,o)
 Plots ICC curves under given item parameters

 INPUT: 
	itemParameters
	o - options

 OUTPUT: 
	res  - figure handle
\end{lstlisting}
%%%%%%%%%%%%%
\par\noindent\rule{\textwidth}{0.4pt}
{\bf\tt deltaScoring.item.MAD}
\par\noindent\rule{\textwidth}{0.4pt}
\begin{lstlisting}[style=Matlab-bw]
 res = MAD(params,observedLogitDelta,o)
 Calculates the Mean Absolute Difference between
 opserved probability for correct response and
 predicted probability obtained under the
 selected RFM model.

 INPUT:
	params             - item parameters
	observedLogitDelta - observed PCR
	o                  - options

 OUTPUT:
 	res - MAD values
\end{lstlisting}
%%%%%%%%%%%%%
\par\noindent\rule{\textwidth}{0.4pt}
{\bf\tt deltaScoring.item.parametersFromCharacteristics}
\par\noindent\rule{\textwidth}{0.4pt}
\begin{lstlisting}[style=Matlab-bw]
 res = parametersFromCharacteristics(location,discrimination,o)
 Calculates the item parameters from item characteristics

 INPUT:
	location
	discrimination
	o           - options

 OUTPUT
	res - [b,s]
\end{lstlisting}
%%%%%%%%%%%%%
\par\noindent\rule{\textwidth}{0.4pt}
{\bf\tt deltaScoring.person.aberrant}
\par\noindent\rule{\textwidth}{0.4pt}
\begin{lstlisting}[style=Matlab-bw]
 res = aberrant(itemParams,itemDeltas,Dscores,itemResponse,options)
 Finds aberrant person behaviour according
 to the "quantile method" according to
 An Examination of Different Methods of Setting Cutoff Values in Person Fit Research
 Amin Mousavi, Ying Cui & Todd Rogers

 INPUT:
	itemParams   - item parameters
	itemDeltas   - item deltas
	Dscores      - persons D-scores
	itemResponse - dichotomous item response

 OUTPUT:
	res - aberrant indicator 0/1
\end{lstlisting}
%%%%%%%%%%%%%
\par\noindent\rule{\textwidth}{0.4pt}
{\bf\tt deltaScoring.person.fitHStatistics}
\par\noindent\rule{\textwidth}{0.4pt}
\begin{lstlisting}[style=Matlab-bw]
 fitHStatistics(item_response)
 Calculates the H statistics for the
 dichotomous item response
\end{lstlisting}
%%%%%%%%%%%%%
\par\noindent\rule{\textwidth}{0.4pt}
{\bf\tt deltaScoring.person.fitIndexZ}
\par\noindent\rule{\textwidth}{0.4pt}
\begin{lstlisting}[style=Matlab-bw]
 fitIndexZ(lklh,Elklh,Vlklk)
  Inputs are from person.likelihood

 based on
 D. Dimitrov, R. Smith. Adjusted Rasch Person-Fit Statistics. J. of
 Applied measurement. 2006
\end{lstlisting}
%%%%%%%%%%%%%
\par\noindent\rule{\textwidth}{0.4pt}
{\bf\tt deltaScoring.person.fitMSE}
\par\noindent\rule{\textwidth}{0.4pt}
\begin{lstlisting}[style=Matlab-bw]
 [Outfit, Infit] = fitMSE(item_response, expected_item_score)
 Calculates the Outfit and Infit of the MSE fit
\end{lstlisting}
%%%%%%%%%%%%%
\par\noindent\rule{\textwidth}{0.4pt}
{\bf\tt deltaScoring.person.fitU}
\par\noindent\rule{\textwidth}{0.4pt}
\begin{lstlisting}[style=Matlab-bw]
 fitU( params, Dscore, responses, o)
 Calculates U statistics
\end{lstlisting}
%%%%%%%%%%%%%
\par\noindent\rule{\textwidth}{0.4pt}
{\bf\tt deltaScoring.person.fitUD1}
\par\noindent\rule{\textwidth}{0.4pt}
\begin{lstlisting}[style=Matlab-bw]
 fitUD1(deltas, Dscore, responses)
 Calculates UD1 Statistics
\end{lstlisting}
%%%%%%%%%%%%%
\par\noindent\rule{\textwidth}{0.4pt}
{\bf\tt deltaScoring.person.fitUD2}
\par\noindent\rule{\textwidth}{0.4pt}
\begin{lstlisting}[style=Matlab-bw]
 fitUD2( deltas, Dscore, responses, params, o)
 Calculates UD2 statistics
\end{lstlisting}
%%%%%%%%%%%%%
\par\noindent\rule{\textwidth}{0.4pt}
{\bf\tt deltaScoring.person.likelihood}
\par\noindent\rule{\textwidth}{0.4pt}
\begin{lstlisting}[style=Matlab-bw]
 [res, expected, variance] = likelihood(dScores,item_parameters,item_response,o)
 Calculates the likelihood for a specific response pattern in item_response from a
 person with ability in dScores, over a set of items in item_parameters.


 based on
 D. Dimitrov, R. Smith. Adjusted Rasch Person-Fit Statistics. J. of
 Applied measurement. 2006

\end{lstlisting}
%%%%%%%%%%%%%
\par\noindent\rule{\textwidth}{0.4pt}
{\bf\tt deltaScoring.person.plotConditionalSE}
\par\noindent\rule{\textwidth}{0.4pt}
\begin{lstlisting}[style=Matlab-bw]
 plotConditionalSE(dScores,SE)
 Plots conditional standard eroor
\end{lstlisting}
%%%%%%%%%%%%%
\par\noindent\rule{\textwidth}{0.4pt}
{\bf\tt deltaScoring.person.responeseProbability}
\par\noindent\rule{\textwidth}{0.4pt}
\begin{lstlisting}[style=Matlab-bw]
 res = responeseProbability(delta,item_parameters,o)
 Calculates the probability for correct response from a
 person with ability delta over the set of items with
 parameters, defined in item_parameters.

 INPUT
   delta - Person ability
   item_parameters - Item parameters [b,s,c]
   o - Delta Scoring Options
\end{lstlisting}
%%%%%%%%%%%%%
\par\noindent\rule{\textwidth}{0.4pt}
{\bf\tt deltaScoring.poly.ccr}
\par\noindent\rule{\textwidth}{0.4pt}
\begin{lstlisting}[style=Matlab-bw]
 ccr(itemParameters,observedLogitDelta,o)
 Plots Cathegory Characteristic Curves for the
 Polytomous items

 INPUT:
   itemParameters - row vector of values of the difficulty parameter
                          for item grades.
	 observedLogitDelta  - person ability value
	 o             - options

\end{lstlisting}
%%%%%%%%%%%%%
\par\noindent\rule{\textwidth}{0.4pt}
{\bf\tt deltaScoring.poly.itemPerformance}
\par\noindent\rule{\textwidth}{0.4pt}
\begin{lstlisting}[style=Matlab-bw]
 res = itemPerformance(itemParameters,delta,o)
Calculates the probability for correct performance for the polytomous item

 INPUT:
   itemParameters - row vector of values of the difficulty parameter
                          for item grades.
	 delta 		   - person ability value
	 o             - options

 OUTPUT:
	res - probability for correct performance
\end{lstlisting}
%%%%%%%%%%%%%
\par\noindent\rule{\textwidth}{0.4pt}
{\bf\tt deltaScoring.poly.logitDeltaPlot}
\par\noindent\rule{\textwidth}{0.4pt}
\begin{lstlisting}[style=Matlab-bw]
 h = logitDeltaPlot(GF,observedLogitDelta,o)
 Plots the fit and the estimated logistics curve
 Returns the figure object

 INPUT:
    GF - output from logitDeltaFit
    observedLogitDelta - from Results of logitDeltaFit
    o - options
           dScale

 OUTPUT:
	h - figure habdle
\end{lstlisting}
%%%%%%%%%%%%%
\par\noindent\rule{\textwidth}{0.4pt}
{\bf\tt deltaScoring.scoring.dScore}
\par\noindent\rule{\textwidth}{0.4pt}
\begin{lstlisting}[style=Matlab-bw]

 Returns the so called d-Score for a person
 with a given response vector over a set
 of items with precalculated deltas;

 INPUT:
	itemDeltas - item delta values
   response   - 0/1 item response
   t          - type
			total / relative_to_n /
           relative_to_d [default]
\end{lstlisting}
%%%%%%%%%%%%%
\par\noindent\rule{\textwidth}{0.4pt}
{\bf\tt deltaScoring.scoring.dScoreSE\_IRT}
\par\noindent\rule{\textwidth}{0.4pt}
\begin{lstlisting}[style=Matlab-bw]
 Returns the so called d-Score SE for a person
 with a given a given ability theta (on a logit scale)
 over a set of items with IRT parameters [b,a,c];

 left here only for convenience.
\end{lstlisting}
%%%%%%%%%%%%%
\par\noindent\rule{\textwidth}{0.4pt}
{\bf\tt deltaScoring.scoring.itemPersonMap}
\par\noindent\rule{\textwidth}{0.4pt}
\begin{lstlisting}[style=Matlab-bw]
\end{lstlisting}
%%%%%%%%%%%%%
\par\noindent\rule{\textwidth}{0.4pt}
{\bf\tt deltaScoring.scoring.observedLogitDelta}
\par\noindent\rule{\textwidth}{0.4pt}
\begin{lstlisting}[style=Matlab-bw]
 observedLogitDelta(ItemResponse, Dscore,o)
 Calculates the proportions of the correct scores
 for different falues of the dScale

 INPUT:
   ItemResponse - dichotomous item response 0/1
   Dscore       - estimated person's dScore
   o            - options
                
 OUTPUT: 
 	res - proportion of correct answers on dScale
\end{lstlisting}
%%%%%%%%%%%%%
\par\noindent\rule{\textwidth}{0.4pt}
{\bf\tt deltaScoring.scoring.Options}
\par\noindent\rule{\textwidth}{0.4pt}
\begin{lstlisting}[style=Matlab-bw]
 option = Options(varargin)
 Defines the options for DELTA SCORING
 Default Values

         NofSamplesForBootstrapping: 1000
   sampleProportionForBootstrapping: 0.1000
            estTypeForBootstrapping: 'mode'
                             dScale: [21x1 double]
                             Models: {1x3 cell}
                         ModelNames: {'RFM1'  'RFM2'  'RFM3'}
                   ModelFixedParams: [1x1 struct]
                 Model_coefficients: {'b'  's'  'c'  'd'}
                              model: 2
                               type: 'raw'
                 skipObservedOnPlot: 2
                   aberrantQuantile: 0.7000
                                 EM: [1x1 struct]
                      StartingPoint: [0.5000 1 0.1000]
                              Lower: [0.0100 0.2000 0]
                              Upper: [0.9900 5 0.5000]
                  RFM_params_method: 'constrained'
\end{lstlisting}
%%%%%%%%%%%%%
\par\noindent\rule{\textwidth}{0.4pt}
{\bf\tt deltaScoring.scoring.PCR}
\par\noindent\rule{\textwidth}{0.4pt}
\begin{lstlisting}[style=Matlab-bw]
 res = PCR(params,delta,o)
 Probability for correct response

   INPUT:
     params - logistics parameres
     delta  - delta values
     o      - options
               mmodel
\end{lstlisting}
%%%%%%%%%%%%%
\par\noindent\rule{\textwidth}{0.4pt}
{\bf\tt deltaScoring.scoring.poly2dih}
\par\noindent\rule{\textwidth}{0.4pt}
\begin{lstlisting}[style=Matlab-bw]
 [DIHscores,Poly,Org] = poly2dih(Response)
 Converts polytomous to dihotomous item respone

 INPUT:
	Response - polytomous item response

 OUTPUT:
	DIHscores - dihotomous ite response
	Poly      - indicator for polytomous items
	Org       - Labels, etc. for poly items
\end{lstlisting}
%%%%%%%%%%%%%
\par\noindent\rule{\textwidth}{0.4pt}
{\bf\tt deltaScoring.scoring.scaledScore}
\par\noindent\rule{\textwidth}{0.4pt}
\begin{lstlisting}[style=Matlab-bw]
 res=scaledScore(scores,t)
 Scales the score according to type t
 t = 'range' scales in range 0..100
\end{lstlisting}
%%%%%%%%%%%%%
\par\noindent\rule{\textwidth}{0.4pt}
{\bf\tt deltaScoring.scoring.trueScore}
\par\noindent\rule{\textwidth}{0.4pt}
\begin{lstlisting}[style=Matlab-bw]
 res = trueScore(itemDeltas, parameters, dScore, o)
 Calculates the true-score measure for person's D-score
 on a set of items with delta scores in itemDeltas,
 logistic parameters of the items and the person's D-score

  INPUT:
   itemDeltas - item's delta scores
   parameters - item's logistics parameters
   dScore    - person's D-score
   o         - Options (defaults scoring.Options)
\end{lstlisting}
%%%%%%%%%%%%%
\par\noindent\rule{\textwidth}{0.4pt}
{\bf\tt deltaScoring.scoring.trueScoreSE}
\par\noindent\rule{\textwidth}{0.4pt}
\begin{lstlisting}[style=Matlab-bw]
 res = trueScoreSE(itemDeltas, parameters, dScore, o)
 Calculates the true-score SE measure for person's D-score
 on a set of items with delta scores in itemDeltas,
 logistic parameters of the items and the person's D-score

  INPUT:
   itemDeltas - item's delta scores
   parameters - item's logistics parameters
   dScore    - person's D-score
   o         - Options (defaults scoring.Options)
\end{lstlisting}


\end{document}